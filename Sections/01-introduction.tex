%*-----------ORIENTADOR -------------------------------------------------------------
\begin{frame}[t]{Horta-bot}
    \transboxout[duration=0.5]
    \framesubtitle{Marco Reis}
    \begin{columns}
        \column{.075\textwidth}
        \column{.4\textwidth}
            \includegraphics[width=.7\textwidth]{darwin-op}
        \column{.55\textwidth}
            \begin{itemize}
                \justifying
                %//todo não use ponto e vírgula, isso não é um relatório é uma apresentação
                %//todo não use bullets
                %//note já enviei o meu short bio para vc, mas...
                %//note Senior Engineer and Researcher with more than 20 years of experience in industrial project management and R&D, including the implementation of two automobile plants in Brazil as well as in steel making, power generation and automation. Develops robotic and manipulator tool designs, autonomous vehicles, asset management, RCM, TPM, reliability and maintenance of critical equipment, and evaluation in FMEA application. In recent 10 years he has worked in development robotic projects.
                \item Possui graduação em Engenharia Elétrica pela Universidade Federal do Paraná (1996) e mestrado em Engenharia de Produção pela Universidade Federal de Santa Catarina (2003);
                % \item 20 DoF\footnote{do inglês, graus de liberdade};
                \item (...);
            \end{itemize}
    \end{columns}
 %*----------- notes
    \note[item]{Notes can help you to remember important information. Turn on the notes option.}
\end{frame}
%-

%*----------- SLIDE -------------------------------------------------------------
\begin{frame}[t]{Justificativa} 
    \transdissolve[duration=0.5]
    %\newline
        \begin{columns}[t]
            \column{.05\linewidth}
            \column{.6\linewidth}
                \begin{itemize}
                    \justifying
                    \item A rotina da vida urbana torna o tempo um recurso escasso, dificultando atividades de cultivo; \cite{G1:3xtransito}
                    \item Há uma longa e complexa cadeia logística na disponibilização de alimentos orgânicos; \cite{silva:cadeiaprodutiva}
                    \item A busca por alimentos orgânicos tem crescido;  \cite{sebrae:organicos}.
                    \item O cultivo de pequenos temperos e hortaliças no ambiente domiciliar favorece uma cultura mais "verde" e participativa; \cite{G1:pequenoagricultor}
                \end{itemize}
            \column{.45\linewidth}
            \begin{center}
            %\centerline{
                \begin{figure}
                    %\includegraphics[width=1\textwidth]{pista}
                    % \caption{Pista de corrida}
                    \hspace{125pt}
                    \roundpic[xshift=0cm,yshift=0cm]{4cm}{5cm}{organic-icon.png}
                    %\caption{Pista de corrida \cite{agostini2007}}
                \end{figure}
            %}
            \end{center}
        \end{columns}
%*----------- notes
    \note[item]{Notes can help you to remember important information. Turn on the notes option.}
\end{frame}
%-

%*----------- SLIDE -------------------------------------------------------------
\begin{frame}
    %\transdissolve[duration=0.5]
    %\hspace*{-1cm}
    \begin{columns}
        %\column{.01\textwidth}
        \column{0.4\textwidth}
            ~\hfill
            \vbox{}\vskip-1.4ex%
            \begin{beamercolorbox}[sep=6em, colsep*=18pt, center, wd=\textwidth, ht=\paperheight]{title page header}%
                \begin{center}
                    \textbf{\huge{PROBLEMA}}\par
                    \vspace*{0.3cm}
                    \textbf{\huge{DE}}\par
                    \vspace*{0.3cm}
                    \textbf{\huge{PESQUISA}}
                \end{center}
            \end{beamercolorbox}%
        \hfill\hfill
        \column{.05\textwidth} 
        \column{.55\textwidth}
        \begin{center}
            %//todo percebo que vc usou forma e forma depois, pense em algo para substituir, fica mais elegante
            \huge{\textbf{"De que forma é possível cultivar hortaliças orgânicas de forma autônoma no ambiente residencial dos centros urbanos?"}}
        \end{center}
        \hfill
       
    \end{columns}
  
 %*----------- notes__
    \note[item]{Notes can help you to remember important information. Turn on the notes option.}
\end{frame}
%-

%*----------- OBJETIVO GERAL -------------------------------------------------------------
\begin{frame}[c]{Objetivo Geral} 
    \transdissolve[duration=0.5]
   
    \begin{center}
        \Wider{%
        \begin{shaded}
        \begin{center}
            \vspace*{0.4cm}
            \resizebox{!}{1.3cm}{%
               % \color{bg} O objetivo é ter um objetivo.
                \begin{tabular}{ccc}
                    %//todo dois objetivos?? ou um só? escolha.
                    Desenvolver um sistema robótico capaz de realizar o \\
                    monitoramento de hortaliças e realizar a plantação\\ 
                    das mesmas.       
                  \end{tabular}
            }%
        \end{center}
        \end{shaded}
        }%
    \end{center}
    
   
%*----------- notes
    \note[item]{Notes can help you to remember important information. Turn on the notes option.}
\end{frame}
%-
%*----------- SLIDE -------------------------------------------------------------
\begin{frame}[t]{Objetivos Específicos}
    \transboxout[duration=0.5]
    \begin{columns}
        \column{.1\textwidth}
        \column{1.8\textwidth}
            \begin{enumerate}
                %//todo não vejo que estes seriam o seus objetivos específicos, eles contribuiem mas acho que deveríam serem lapidados.
                \item Realizar o estudo do estado da arte;
                \item Desenvolver sistema de movimentação para a unidade de supervisionamento;
                \item Desenvolver sistema de monitoramento visual (computacional);
                \item Desenvolver sistema de atuação para plantação e aquecimento;
                \item Desenvolver página de informações para o compartilhamento dos resultados com 
                \\ os usuários;
                \item Realizar demonstração do sistema;
                \item Desenvolvimento de artigos científicos.

            \end{enumerate}
    \end{columns}
 %*----------- notes
    \note[item]{Notes can help you to remember important information. Turn on the notes option.}
\end{frame}
%-

%*----------- SLIDE -------------------------------------------------------------
\begin{frame}[c]{Metodologia}
    %\transboxin[duration=1,direction=30]

    \includegraphics[width=1.04\textwidth]{metodologia.png}

%*----------- notes
    \note[item]{Notes can help you to remember important information. Turn on the notes option.}
\end{frame}
%-
